% Template for ICASSP-2018 paper; to be used with:
%          spconf.sty  - ICASSP/ICIP LaTeX style file, and
%          IEEEbib.bst - IEEE bibliography style file.
% --------------------------------------------------------------------------
\documentclass{article}
\usepackage{spconf,amsmath,graphicx,euler,amsbsy}
%\usepackage[mathscr]{eucal}






% Commande de correc à affiner
\newcommand{\rev}[1]{{\color{black}#1}}
\newcommand{\modif}[1]{{\color{black}#1}}
\newcommand{\modifR}[1]{{\color{black}#1}}
\newcommand{\revv}[1]{{\color{black}#1}}


% Example definitions.
% --------------------
% utils
\def\F{{\mathcal F}}
% list of tensor used  in text
\def\tA{\boldsymbol{{\mathscr A}}}
\def\X{\boldsymbol{{\mathscr X}}}
\def\G{\boldsymbol{{\mathscr G}}}
\def\C{\boldsymbol{{\mathscr C}}}
% list of matrix used in text
\def\A{{\mathbf A}}
\def\B{{\mathbf B}}


% Title.
% ------
\title{PAPER ICASSP}
%
% Single address.
% ---------------
\name{Author(s) Name(s)\thanks{Thanks to DAISY Normandy region for funding.}}
\address{Author Affiliation(s)}
%
% For example:
% ------------
%\address{School\\
%	Department\\
%	Address}
%
% Two addresses (uncomment and modify for two-address case).
% ----------------------------------------------------------
%\twoauthors
%  {A. Author-one, B. Author-two\sthanks{Thanks to XYZ agency for funding.}}
%	{School A-B\\
%	Department A-B\\
%	Address A-B}
%  {C. Author-three, D. Author-four\sthanks{The fourth author performed the work
%	while at ...}}
%	{School C-D\\
%	Department C-D\\
%	Address C-D}
%
\begin{document}
%\ninept
%
\maketitle
%
\begin{abstract}
abstract here
\end{abstract}
%
\begin{keywords}
One, two, three, four, five
\end{keywords}
%
\section{Introduction}
\label{sec:intro}


\subsection*{Notation}
\label{ssec:notation}

Vectors are denoted by boldface lowercase letters, e.g., $\mathbf{a}$. Matrices are denoted by boldface capital letters, e.g., $\A$. Higher-order
tensors are denoted by boldface Euler script letters, e.g., $\tA$.
Scalars are denoted by lowercase letters, e.g., $a$.


\section{Non-negative Tucker decomposition}
\label{ssec:subhead}

$$
\| \X - \G \times_1 \A_s \times_2 \A_f  \times_3 \A_t \|_{\F}^2 
$$

regularisation
$$
\| \C - \G \times_1 \B_s \times_2 \B_f  \times_3 \B_t \|_{\F}^2
$$

 
\subsubsection{Sub-subheadings}
\label{sssec:subsubhead}





% Below is an example of how to insert images. Delete the ``\vspace'' line,
% uncomment the preceding line ``\centerline...'' and replace ``imageX.ps''
% with a suitable PostScript file name.
% -------------------------------------------------------------------------
%\begin{figure}[htb]
%
%\begin{minipage}[b]{1.0\linewidth}
%  \centering
%  \centerline{\includegraphics[width=8.5cm]{image1}}
%%  \vspace{2.0cm}
%  \centerline{(a) Result 1}\medskip
%\end{minipage}
%%
%\begin{minipage}[b]{.48\linewidth}
%  \centering
%  \centerline{\includegraphics[width=4.0cm]{image3}}
%%  \vspace{1.5cm}
%  \centerline{(b) Results 3}\medskip
%\end{minipage}
%\hfill
%\begin{minipage}[b]{0.48\linewidth}
%  \centering
%  \centerline{\includegraphics[width=4.0cm]{image4}}
%%  \vspace{1.5cm}
%  \centerline{(c) Result 4}\medskip
%\end{minipage}
%%
%\caption{Example of placing a figure with experimental results.}
%\label{fig:res}
%%
%\end{figure}




\section{REFERENCES}
\label{sec:refs}


% References should be produced using the bibtex program from suitable
% BiBTeX files (here: strings, refs, manuals). The IEEEbib.bst bibliography
% style file from IEEE produces unsorted bibliography list.
% -------------------------------------------------------------------------
\bibliographystyle{IEEEbib}
\bibliography{strings,refs}

\end{document}
